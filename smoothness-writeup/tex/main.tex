%%%%%%%%%
% FLAGS %
%%%%%%%%%

% Is this an internal version? (i.e., not public-facing.)
\ifdefined \isinternal \else \def \isinternal{1} \fi

% Is this an extended version? (e.g., includes appendix.)
\ifdefined \isextended \else \def \isextended{1} \fi

% Should authors be identified? (e.g., not submitted for blind peer review.)
\ifdefined \isauthorid \else \def \isauthorid{1} \fi

% Defaults can be overridden from the Makefile
\newif \ifinternal \if \isinternal 0 \internalfalse \else \internaltrue \fi
\newif \ifextended \if \isextended 0 \extendedfalse \else \extendedtrue \fi
\newif \ifauthorid \if \isauthorid 0 \authoridfalse \else \authoridtrue \fi

%%%%%%%%%%%%
% DOCUMENT %
%%%%%%%%%%%%

\documentclass{article}

\usepackage[T5,T1]{fontenc}

\usepackage{longtable}

\usepackage{microtype}                % Better typesetting and layout algorithms

\usepackage{float}                    % Float layout placement 
\usepackage{placeins}                 % and management 

\usepackage{hyperref}                 % Hyperlinked references
\hypersetup{
    colorlinks,
    linkcolor={red!50!black},
    citecolor={blue!50!black},
    urlcolor={blue!80!black}
}

\usepackage{amsmath}                  % AMS math tools
\usepackage{amsfonts}                 % ...
\usepackage{amssymb}                  % ...
\usepackage{amsthm}                   % ...
\usepackage{mathtools}                % ...

\usepackage{cleveref}

\newtheorem{fact}{Fact}[section]                % Standard theorem environments
\newtheorem*{fact*}{Fact}                       % ...
\newtheorem{definition}{Definition}[section]    % ...
\newtheorem*{definition*}{Definition}           % ...
\newtheorem{proposition}{Proposition}[section]  % ...
\newtheorem*{proposition*}{Proposition}         % ...
\newtheorem{theorem}{Theorem}[section]          % ...
\newtheorem*{theorem*}{Theorem}                 % ...
\newtheorem{lemma}[theorem]{Lemma}              % ...
\newtheorem*{lemma*}{Lemma}                     % ...
\newtheorem{sublemma}[theorem]{Sublemma}        % ...
\newtheorem*{sublemma*}{Sublemma}               % ...
\newtheorem{corollary}{Corollary}[theorem]      % ...
\newtheorem*{corollary*}{Corollary}             % ...

\usepackage[dvipsnames]{xcolor}       % Colors

\usepackage{natbib}                   % Better bibs

\usepackage{stmaryrd}                 % More symbols
\usepackage{pifont}                   % ...
\usepackage{textcomp}                 % ...

\usepackage{centernot}                % slashing things

\usepackage{stackengine}              % Symbol Stacking
\stackMath

\usepackage{nicefrac}                 % Nice Fractions

\usepackage{array}                    % Array Layout
\usepackage{tabularx}                 % Improved Tabular Layout

% Remove default column spacing; set to 0

\setlength\arraycolsep{0pt}           % default is 6pt
\renewcommand\arraystretch{1}         % default is 1

\usepackage{framed}                   % Framed Environment

\usepackage{lipsum}                   % Easy placeholder text generation

\usepackage{enumitem}                 % Better control over itemize/enumerate

\usepackage{mathpartir}               % Flexible math display layout

\usepackage{galois}                   % Galois connections

\usepackage{geometry}                 % Margins etc. (define this locally)

\usepackage{tikz}

\usepackage{relsize}

\newcommand{\shortnearrow}{\mathrel{\rotatebox[origin=c]{33}{$\rightarrow$}}}
\newcommand{\shortsearrow}{\mathrel{\rotatebox[origin=c]{-33}{$\rightarrow$}}}

\newcommand{\slashedrel}[1]{\mathrel{\centernot{#1}}}

\ifinternal
  \newcommand{\note}[1]{{\color{red}#1}}
  \newcommand{\todo}[1]{\note{TODO: #1}}
\else
  \newcommand{\note}[1]{}
  \newcommand{\todo}[1]{}
\fi


\newcommand*\circled[1]{\textcircled{\raisebox{-0.1pt}{\scalebox{0.66}{#1}}}}

\newcommand{\mtext}[1]{\ifmmode\operatorname{#1}\else\textnormal{#1}\fi}
\newcommand{\mtexttt}[1]{\ifmmode\operatorname{\mathtt{#1}}\else\textnormal{\texttt{#1}}\fi}
\newcommand{\mtextit}[1]{\ifmmode\operatorname{\mathit{#1}}\else\textnormal{\textit{#1}}\fi}
\newcommand{\mtextbf}[1]{\ifmmode\operatorname{\mathbf{#1}}\else\textnormal{\textbf{#1}}\fi}
\newcommand{\mtextsc}[1]{\ifmmode\operatorname{\textsc{\smaller #1}}\else\textnormal{\textsc{\smaller #1}}\fi}

% colors %

\definecolor{cbsafeABright}{RGB}{8,72,145}
\definecolor{cbsafeADark}{RGB}{109,36,150}

\definecolor{cbsafeBBright}{RGB}{85,119,13}
\definecolor{cbsafeBDark}{RGB}{109,70,17}

\definecolor{cbsafeCBright}{RGB}{150,48,89}
\definecolor{cbsafeCDark}{RGB}{70,24,48}

\newcommand{\colorMATHA}{cbsafeABright}
\newcommand{\colorSYNTAXA}{cbsafeADark!80!black}
\newcommand{\colorMATHB}{cbsafeBBright}
\newcommand{\colorSYNTAXB}{cbsafeBDark}
\newcommand{\colorMATHC}{cbsafeCBright}
\newcommand{\colorSYNTAXC}{cbsafeCDark}

\newcommand{\colorTEXT}{black}

\newcommand{\colorMATH}{\colorMATHA}
\newcommand{\colorSYNTAX}{\colorSYNTAXA}



\setlength{\parindent}{0pt}
\setlength{\parskip}{8pt}

\begin{document}

\section{A Tiny Model (No Vectors)}

A single layer neural net implements essentially just a single multiplication.

We have an input vector ⸨𝐱 ≜ [x₁,x₂]⸩, a weights vector ⸨𝛉 ≜ [θ₁,θ₂]⸩, and the
classification function ⸨f⸩ is:
M⁅ f⸤𝛉⸥(𝐱) ≜ θ₁x₁ + θ₂x₂ M⁆

Here is the mean squared error loss function over scalars:
M⁅ L(a,b) ≜ (a - b)² M⁆
and the loss of our function ⸨f⸩ {w.r.t.} a training example ⸨y⸩ is:
M⁅ L⸤f⸤𝛉⸥⸥(𝐱,y) Aː[t]rcl
                A⁅ ⧼≜⧽ (f⸤𝛉⸥(𝐱) - y)² 
                A⁃ ⧼=⧽ (θ₁x₁ + θ₂x₂ - y)²
                A⁃ ⧼=⧽ θ₁²x₁² + θ₂²x₂² + 2θ₁θ₂x₁x₂ - 2θ₁x₁y - 2θ₂x₂y + y²
                A⁆
M⁆
Here is the gradient of ⸨L⸤f⸤𝛉⸥⸥(𝐱,y)⸩ {w.r.t.} ⸨𝛉⸩:
M⁅ ∇⸤𝛉⸥L⸤f⸤𝛉⸥⸥(𝐱,y) Aː[t]rcl
                    A⁅ ⧼≜⧽ ‘[ \frac{∂(L⸤f⸤𝛉⸥⸥(𝐱,y))}{∂θ₁} , \frac{∂(L⸤f⸤𝛉⸥⸥(𝐱,y))}{∂θ₂} ’]
                    A⁃ ⧼=⧽ ‘[ 2θ₁x₁²+ 2θ₂x₁x₂ - 2x₁y , 2θ₂x₂²+ 2θ₁x₁x₂ - 2x₂y ’]
                    A⁆
M⁆
The «immediate sensitivity» is defined as:
M⁅ «𝐈𝐒»⸤𝐱,𝛉⸥(f) Aː[t]rcl
                  A⁅ ⧼≜⧽ ‖ ∇⸤𝐱⸥ ‖ ∇⸤𝛉⸥ L⸤f⸤𝛉⸥⸥(𝐱,y) ‖₂ ‖₂
                  A⁃ ⧼=⧽ ‖ ∇⸤𝐱⸥ ‖ ‘[ 2θ₁x₁²+ 2θ₂x₁x₂ - 2x₁y , 2θ₂x₂²+ 2θ₁x₁x₂ - 2x₂y ’] ‖₂ ‖₂
                  A⁃ ⧼=⧽ ‖ ∇⸤𝐱⸥ √{(2θ₁x₁²+ 2θ₂x₁x₂ - 2x₁y)² + (2θ₂x₂²+ 2θ₁x₁x₂ - 2x₂y)²} ‖₂
                  A⁃ ⧼=⧽ ‖ ∇⸤𝐱⸥ 2√{(x₁² + x₂²)(θ₁x₁ + θ₂x₂ - y)²} ‖₂
                  A⁆
M⁆
% We want to know ⸨β⸩ such that the above quantity is ⸨β⸩-smooth.
% The derivative of the loss is:
% M⁅ \frac{∂L⸤f⸤θ⸥⸥(x,y)}{∂θᵀ} Aː[t]rcl
%                              A⁅ ⧼≜⧽ \frac{∂(xθᵀ - y)²}{∂θᵀ}
%                              A⁃ ⧼=⧽ 2(xθᵀ - y)\frac{∂(xθᵀ - y)}{∂θᵀ}
%                              A⁃ ⧼=⧽ 2(xθᵀ - y)(\frac{∂(xθᵀ)}{∂θᵀ} - \frac{∂y}{∂θᵀ})
%                              A⁃ ⧼=⧽ 2(xθᵀ - y)\frac{∂(xθᵀ)}{∂θᵀ}
%                              A⁃ ⧼=⧽ 2x(xθᵀ - y)\frac{∂θᵀ}{∂θᵀ}
%                              A⁃ ⧼=⧽ 2x(xθᵀ - y)
%                              A⁆
% M⁃
% M⁃ \frac{∂L⸤f⸤θ⸥⸥(x,y)}{∂θᵀ} Aː[t]rcl
%                              A⁅ ⧼≜⧽ \frac{∂(xθᵀ - y)²}{∂θᵀ}
%                              A⁃ ⧼=⧽ \frac{∂((xθᵀ)² - 2xyθᵀ - y²)}{∂θᵀ}
%                              A⁃ ⧼=⧽ \frac{∂((xθᵀ)²)}{∂θᵀ} - \frac{∂(2xyθᵀ)}{∂θᵀ} - \frac{∂(y²)}{∂θᵀ}
%                              A⁃ ⧼=⧽ \frac{∂((xθᵀ)²)}{∂θᵀ} - 2xy\frac{∂θᵀ}{∂θ}
%                              A⁃ ⧼=⧽ 2x²θᵀ\frac{∂θᵀ}{∂θᵀ} - 2xy\frac{∂θᵀ}{∂θᵀ}
%                              A⁃ ⧼=⧽ 2x²θᵀ - 2xy
%                              A⁃ ⧼=⧽ 2x(xθᵀ - y)
%                              A⁃ ⧼=⧽ ... 2xᵀ(xθᵀ - y)
%                              A⁆
% M⁆

\section{A Tiny Model (Vectorized)}

A single layer neural net implements essentially just a single multiplication.

We have an input vector ⸨x⸩, a weights vector ⸨θ⸩, and the
classification function ⸨f⸩ is:
M⁅ f⸤θ⸥(x) ≜ xθᵀ M⁆

Here is the mean squared error loss function over scalars:
M⁅ L(a,b) ≜ (a - b)² M⁆
and the loss of our function ⸨f⸩ {w.r.t.} a training example ⸨y⸩ is:
M⁅ L⸤f⸤θ⸥⸥(x,y) Aː[t]rcl
                A⁅ ⧼≜⧽ (f⸤θ⸥(x) - y)² 
                A⁃ ⧼=⧽ (xθᵀ - y)²
                A⁆
M⁆
The «immediate sensitivity» is defined as:
M⁅ ‖ ∇ₓ ‖ ∇⸤θ⸥ L⸤f⸤θ⸥⸥(x,y) ‖₂ ‖₂
M⁆
We want to know ⸨β⸩ such that the above quantity is ⸨β⸩-smooth.
The derivative of the loss is:
M⁅ \frac{∂L⸤f⸤θ⸥⸥(x,y)}{∂θᵀ} Aː[t]rcl
                             A⁅ ⧼≜⧽ \frac{∂(xθᵀ - y)²}{∂θᵀ}
                             A⁃ ⧼=⧽ 2(xθᵀ - y)\frac{∂(xθᵀ - y)}{∂θᵀ}
                             A⁃ ⧼=⧽ 2(xθᵀ - y)(\frac{∂(xθᵀ)}{∂θᵀ} - \frac{∂y}{∂θᵀ})
                             A⁃ ⧼=⧽ 2(xθᵀ - y)\frac{∂(xθᵀ)}{∂θᵀ}
                             A⁃ ⧼=⧽ 2x(xθᵀ - y)\frac{∂θᵀ}{∂θᵀ}
                             A⁃ ⧼=⧽ 2x(xθᵀ - y)
                             A⁆
M⁃
M⁃ \frac{∂L⸤f⸤θ⸥⸥(x,y)}{∂θᵀ} Aː[t]rcl
                             A⁅ ⧼≜⧽ \frac{∂(xθᵀ - y)²}{∂θᵀ}
                             A⁃ ⧼=⧽ \frac{∂((xθᵀ)² - 2xyθᵀ - y²)}{∂θᵀ}
                             A⁃ ⧼=⧽ \frac{∂((xθᵀ)²)}{∂θᵀ} - \frac{∂(2xyθᵀ)}{∂θᵀ} - \frac{∂(y²)}{∂θᵀ}
                             A⁃ ⧼=⧽ \frac{∂((xθᵀ)²)}{∂θᵀ} - 2xy\frac{∂θᵀ}{∂θ}
                             A⁃ ⧼=⧽ 2x²θᵀ\frac{∂θᵀ}{∂θᵀ} - 2xy\frac{∂θᵀ}{∂θᵀ}
                             A⁃ ⧼=⧽ 2x²θᵀ - 2xy
                             A⁃ ⧼=⧽ 2x(xθᵀ - y)
                             A⁃ ⧼=⧽ ... 2xᵀ(xθᵀ - y)
                             A⁆
M⁆

\section{New}

% Need: IS is ⸨β⸩-smooth.

% Know: IS(x)
% Need to know: IS is ⸨β⸩-smooth
% Need to know: ⸨ | IS(x) - IS(x') | ≤ ???⸩

% If IS is ⸨β⸩-smooth then ⸨| IS(x) - IS(x') | ≤ β⸩

% We care about: local sensitivity of gradient G

% ‖ G(x) - G(x') ‖₂ ≤ IS(x) + β

% if G' is β-smooth then:

% ‖G'(x) - G'(x')‖ ≤ β ‖x - x'‖        (defn of smoothness)
% ‖G'(x) - G'(x')‖ ≤ β                 (‖x - x'‖ ≤ 1 by assumption)


% LS⸤G⸥(x) = max⸤x' . d(x, x') ≤ 1⸥ ‖ G(x) - G(x') ‖

% MVT says:
% ∀ x, x'. ∃ x''. x ≤ x'' ≤ x' ∧ ‖G(x) - G(x')‖ / ‖x - x'‖ = G'(x'')

%    ‖G(x) - G(x')‖ / ‖x-x'‖ = G'(x'') for some x''           (mean value theorem)
% => ‖G(x) - G(x')‖ / ‖x-x'‖ = G'(x) + ‖G'(x) - G'(x'')‖      (arithmetic)
% => ‖G(x) - G(x')‖ / ‖x-x'‖ ≤ G'(x) + β ‖x - x''‖            (def. of β-smoothness)
% => ‖G(x) - G(x')‖ / ‖x-x'‖ ≤ G'(x) + β                      (def. absolute value)
% => ‖G(x) - G(x')‖          ≤ G'(x) + β                      (inequality)
% => LS⸤G⸥(x)                ≤ G'(x) + β                      (def. of LS)



% => G(x) - G(x') = G'(x) + ‖G'(x) - G'(x'')‖ 
% => G(x) - G(x') ≤ G'(x) + β                          (by above)
% => LS⸤G⸥(x)     ≤ G'(x) + β                          (def of LS)



\section{Current Situation}

For all of the following, ⸨‖X‖₁⸩ is the L1 norm, ⸨‖X‖₂⸩ is the L2 norm, ⸨‖X‖⸤∞⸥⸩
is the L❪∞❫ norm, and ⸨‖X‖⸩ (without a subscript) is some norm in a parametrized
metric space. Same goes for distances ⸨‖X-Y‖₁⸩, ⸨‖X-Y‖₂⸩, ⸨‖X-Y‖⸤∞⸥⸩ and
⸨‖X-Y‖⸩.

\subsection{From \cite{smooth-sensitivity}}

\begin{definition}[Global Sensitivity \citep{smooth-sensitivity}]
  The Global Sensitivity of ⸨f⸩ is ⸨«GS»⸤f⸥⸩ where:
  M⁅ X⁅ ⩊ «GS»⸤f⸥ ≜ \max\limits⸤x,y:‖x-y‖=1⸥ ‖ f(x) - f(y) ‖₁ 
        ⩊ ⟪\citep[§ 1.2, Definition 1.3]{smooth-sensitivity}⟫
     X⁆
  M⁆
  Note: ⸨‖x-y‖⸩ is an abstract distance but ⸨‖ f(x) - f(y) ‖₁⸩  is
  specifically L1 distance.
\end{definition}

\begin{definition}[Local Sensitivity \citep{smooth-sensitivity}]
  The Local Sensitivity of ⸨f⸩ at ⸨x⸩ is ⸨«LS»⸤f⸥(x)⸩ where:
  M⁅ X⁅ ⩊ «LS»⸤f⸥(x) ≜ \max\limits⸤y:‖x-y‖=1⸥ ‖ f(x) - f(y) ‖₁
        ⩊ ⟪\citep[§ 1.3, Definition 1.6]{smooth-sensitivity}⟫
     X⁆
  M⁆
  Note: ⸨‖x-y‖⸩ is an abstract distance but ⸨‖ f(x) - f(y) ‖₁⸩  is
  specifically L1 distance.
\end{definition}

\begin{definition}[DP-⸨β⸩-Smoothness \citep{smooth-sensitivity}]\ \\
  A function ⸨f⸩ is DP-⸨β⸩-Smooth if:
  M⁅ X⁅ ⩊ «GS»⸤\ln(f(⋅))⸥ ≤ β
        ⩊ ⟪\citep[§ 2.1]{smooth-sensitivity}⟫
     X⁆
  M⁆
  Alternatively, ⸨f⸩ is DP-⸨β⸩-smooth if:
  M⁅ X⁅ ⩊ ∀ x,y⍪ ‖x-y‖ = 1 ⟹ f(x) ≤ e⸢β⸣f(y)
        ⩊ ⟪\citep[§ 2.1, Definition 2.1, (2)]{smooth-sensitivity}⟫
     X⁆
  M⁆
  Note: \cite{smooth-sensitivity} just calls this “⸨β⸩-smooth”.
  \subparagraph{Why are these the same?} 
  ⟪
  M⁅ Aːlcl@{␠}l
     A⁅ ⧼ ⧽ «GS»⸤\ln(f(⋅))⸥ ≤ β
     A⁃ ⧼⟺⧽ ∀ x,y⍪ ‖x-y‖ = 1 ⇒ ‖\ln(f(x)) - \ln(f(y))‖₁ ≤ β & ⟅definition⟆
     A⁃ ⧼⟺⧽ ∀ x,y⍪ ‖x-y‖ = 1 ⇒ \ln(f(x)) ≤ β + \ln(f(y))    & ⟅(somehow?)⟆
     A⁃ ⧼⟺⧽ ∀ x,y⍪ ‖x-y‖ = 1 ⇒ f(x) ≤ e⸢β⸣f(y)              & ⟅algebra⟆
     A⁆
  M⁆
  ⟫
\end{definition}

\begin{definition}[DP-⸨β⸩-Smooth Sensitivity \citep{smooth-sensitivity}]\ \\
  A function ⸨f⸩ has DP-⸨β⸩-Smooth sensitivity if its local sensitivity is
  ⸨β⸩-smooth:
  M⁅ «GS»⸤\ln(«LS»⸤f⸥(⋅))⸥ ≤ β M⁆
  or:
  M⁅ ∀ x,y⍪ ‖x-y‖ = 1 ⟹ «LS»⸤f⸥(x) ≤ e⸢β⸣«LS»⸤f⸥(y) M⁆
  \subparagraph{What if ⸨«LS»⸤f⸥⸩ isn't already smooth?}
  ⟪
  \cite{smooth-sensitivity} defines a construction of ⸨S⸢*⸣⸤f,β⸥⸩
  which is the best upper bound on the local sensitivity of ⸨f⸩ which is also
  ⸨β⸩-smooth; this construction is:
  M⁅ S⸢*⸣⸤f,β⸥(x) = \max⸤y⸥«LS»⸤f⸥(y)e⸢-β‖x-y‖⸣ M⁆
  ⟫
\end{definition}

\subsection{From \cite{metrics-local-sensitivity}}

\begin{definition}[Generalized Global Sensitivity \citep{metrics-local-sensitivity}]
  The Generalized Global Sensitivity of ⸨f⸩ is ⸨«GGS»⸤f⸥⸩ where:
  M⁅ X⁅ ⩊ «GGS»⸤f⸥ ≜ \max⸤x,y⸥ \frac{‖f(x)-f(y)‖}{‖x-y‖}
        ⩊ ⟪\citep[§ 3.2, Definition 2]{metrics-local-sensitivity}⟫
     X⁆
  M⁆
  Note: \cite{metrics-local-sensitivity} just calls this “Global Sensitivity”
\end{definition}

\begin{definition}[Derivative Sensitivity \citep{metrics-local-sensitivity}]
  The Derivative Sensitivity of ⸨f⸩ at ⸨x⸩ is ⸨«DS»⸤f⸥(x)⸩ where:
  M⁅ X⁅ ⩊ «DS»⸤f⸥(x) ≜ ‖f′(x)‖
        ⩊ ⟪\citep[§ 4.2, Definition 12]{metrics-local-sensitivity}⟫
     X⁆
  M⁆
\end{definition}

\paragraph{From ML literature}

\begin{definition}[ML-⸨β⸩-Smoothness]
  A function ⸨f⸩ is ML-⸨β⸩-Smooth if:
  M⁅ X⁅ ⩊ ‖f′(x) - f′(y)‖ ≤ β‖x-y‖ 
        ⩊ ⟪\citep[§ 3]{convex-optimization-ml-slides}⟫
     X⁆
  M⁆
  or:
  M⁅ «GGS»⸤f′⸥ ≤ β M⁆
\end{definition}

\subsection{New Definitions}

\begin{definition}[Immediate Sensitivity]
  The Immediate Sensitivity of ⸨f⸩ at ⸨x⸩ is ⸨«IM»⸤f⸥(x)⸩ where:
  M⁅ «IS»⸤f⸥(x) ≜ ‖f′(x)‖ M⁆
  J⁅ Note: ⸨«IS»⸤f⸥(x) = «DS»⸤f⸥(x)⸩.
  J⁃ Note: In our case studies we pick specifically the L2 norm, so in that
     setting, ⸨«IS»⸤f⸥(x) ≜ ‖f′(x)‖₂⸩ and is an instantiation of ⸨«DS»⸤f⸥(x)⸩
     for a particular norm.
  J⁆
\end{definition}

\subsection{Where we are}

We have observed that for some architectures, using the L1 norm, ⸨d«IS»⸤f⸥/dx⸩
is constant, i.e.,  ⸨d«IS»⸤f⸥/dx = C⸩ for some scalar ⸨C⸩. So ⸨«IS»⸤f⸥⸩ is
ML-⸨β⸩-smooth for ⸨β=0⸩, and ⸨f⸩ is ML-⸨β⸩-smooth for ⸨β=C⸩.

We are currently investigating the following questions:
E⁅ Given an ML-⸨β⸩-smooth bound on ⸨f⸩, can we construct some DP-⸨β⸩-smooth bound
   on ⸨f⸩?
E⁃ Given an ML-⸨β⸩-smooth bound on some altered ⸨f⸩ (e.g., ⸨\ln(f)⸩), can we
   construct some DP-⸨β⸩-smooth bound on ⸨f⸩?
E⁃ Given that ⸨d«IS»⸤f⸥/dx⸩ is constant (or even better; it's ⸨0⸩) for our ⸨f⸩
   in question, can we directly say something about ⸨«IS»⸤f⸥⸩ being DP-⸨β⸩-smooth?
E⁆

\section{Bounded Global Sensitivity Analysis}

Environment ⸨γ⸩ encodes upper and lower bound assumptions on input variables.
E.g., ⸨γ(x) = -1,1⸩ encodes that ⸨x⸩ is guaranteed to range between ⸨-1⸩ and
⸨1⸩. This is exactly the interval abstract domain from the literature on
abstract interpretation.

The function ⸨⟦‗⟧⸩ is the algorithm for bounded global sensitivity analysis. It
takes as input an expression and an environment ⸨γ⸩ and returns a mapping from
variables ⸨x ∈ ‹var›⸩ to a pair of «sensitivity ranges» ⸨sˡ,sʰ⸩, and a pair of
«value ranges» ⸨rˡ,rʰ⸩.

A couple of notes:
I⁅ Most global sensitivity analyses assume positive real numbers as values. We
   must relax this to negative reals in order for ⦑relu⦒ to make sense. A lot
   what is going on (e.g., in the ⸨×⸩ definition) is accounting for the presence
   of negative reals, e.g., multiplying two large negative-valued lower bounds
   could result in a positive-valued upper bound. Sensitivities are also tracked
   with their sign, e.g., the expression ⸨-1x⸩ is ⸨-1⸩-sensitive in ⸨x⸩. Because
   sensitivities are now signed, we must track a range of sensitivities.
I⁃ We could dramatically simplify things by operating over non-negative reals
   and using some hypothetical alternative to relu, but I get the feeling
   supporting negative reals will eventually be essential for ML applications,
   so I've pressed ahead and complicated things with negative reals.
I⁃ The log operation “in math” is only defined for positive valued reals, so
   I've the operation I encode is instead “log of absolute value” so it can be
   defined on all reals. This is why the join or meet is taken of absolute value
   lower and upper bounds in the value range. I'm unsure if the numerators
   should be ⸨sˡ⸩ and ⸨sʰ⸩ as written, or if we should also be taking ⸨sˡ⊓sʰ⸩
   and ⸨sˡ⊔sʰ⸩ respectively. (We should work out some small examples…)
I⁃ In the relu operation's definition, I'm similarly unsure about the treatment
   of ⸨sˡ⸩ and ⸨sʰ⸩ in the definitions for ⸨sˡ′⸩ and ⸨sʰ′⸩.
I⁆

M⁅ 
X⁅ ⟪«Setup (syntactic categories, etc.)…»⟫ ⩊
X⁃
X⁃ Aːrclcl
   A⁅ n   ⧼∈⧽ ℕ
   A⁃ r   ⧼∈⧽ ℝ
   % A⁃ ṙ   ⧼∈⧽ ℝ⸢∞⸣  ⧼≜⧽ ℝ ⊎ ❴-∞,∞❵
   A⁃ s   ⧼∈⧽ ℝ⁺    ⧼≜⧽ ❴r ¦ 0 ≤ r❵
   A⁃ x   ⧼∈⧽ ‹var›
   A⁃ e   ⧼∈⧽ ‹exp› ⧼⩴⧽ r ¦ x ¦ e+e ¦ e×e ¦ ㏑e ¦ ⦑relu⦒(e)
   A⁃ γ   ⧼∈⧽ ‹env› ⧼≜⧽ ‹var› → ℝ × ℝ
   A⁆
X⁃
X⁃ ⟪«Bounded Global Sensitivity Algorithm…»⟫ ⩊
X⁃
X⁃ Aːrcl
   A⁅ ⟦‗⟧⸢‗⸣            ⧼∈⧽ ‹exp› × (‹var› → ℝ × ℝ) → (‹var› ⇀ ℝ⁺ × (ℝ × ℝ))
   A⁃ ⟦r⟧⸢γ⸣(x)         ⧼≜⧽ 0,(r,r)
   A⁃ ⟦x⟧⸢γ⸣(y)         ⧼≜⧽ ‘❴ Aːl@{␠}c@{␠}l
                               A⁅ 1,γ(y) ⧼⟪«if»⟫⧽ x = y
                               A⁃ 0,γ(y) ⧼⟪«if»⟫⧽ x ≠ y
                               A⁆ ’.
   A⁃ ⟦e₁+e₂⟧⸢γ⸣(x)     ⧼≜⧽ Aː[t]l
                            A⁅ s,(rˡ,rʰ)
                            A⁃ ␠⟪«where»⟫ ␠ Aː[t]rcl
                                            A⁅ s₁,(rˡ₁,rʰ₁) ⧼=⧽ ⟦e₁⟧⸢γ⸣(x)
                                            A⁃ s₂,(rˡ₂,rʰ₂) ⧼=⧽ ⟦e₂⟧⸢γ⸣(x)
                                            A⁃ s            ⧼=⧽ s₁ + s₂
                                            A⁃ rˡ           ⧼=⧽ rˡ₁ + rˡ₂
                                            A⁃ rʰ           ⧼=⧽ rʰ₁ + rʰ₂
                                            A⁆
                            A⁆
   A⁃ ⟦e₁×e₂⟧⸢γ⸣(x)     ⧼≜⧽ Aː[t]l
                            A⁅ s,(rˡ,rʰ)
                            A⁃ ␠⟪«where»⟫ ␠ Aː[t]rcl
                                            A⁅ s₁,(rˡ₁,rʰ₁) ⧼=⧽ ⟦e₁⟧⸢γ⸣(x)
                                            A⁃ s₂,(rˡ₂,rʰ₂) ⧼=⧽ ⟦e₂⟧⸢γ⸣(x)
                                            A⁃ s ⧼=⧽ s₁(|rˡ₂| ⊔ |rʰ₂|) + s₂(|rˡ₁| ⊔ |rʰ₁|)
                                            A⁃ rˡ ⧼=⧽ rˡ₁rˡ₂ ⊓ rˡ₁rʰ₂ ⊓ rʰ₁rˡ₂ ⊓ rʰ₁rʰ₂
                                            A⁃ rʰ ⧼=⧽ rˡ₁rˡ₂ ⊔ rˡ₁rʰ₂ ⊔ rʰ₁rˡ₂ ⊔ rʰ₁rʰ₂
                                            A⁆
                            A⁆
   A⁃ ⟦eⁿ⟧                ⧼≜⧽ Aː[t]l
                              A⁅ s′,(rˡ′,rʰ′)
                              A⁃ ␠⟪«where»⟫ ␠ Aː[t]rcl
                                              A⁅ s,(rˡ,rʰ) ⧼=⧽ ⟦e⟧⸢γ⸣(x)
                                              A⁃ s′ ⧼=⧽ … n⟨e⟩⸢n-1⸣ s
                                              A⁃ rˡ′ ⧼=⧽ ???
                                              A⁃ rʰ′ ⧼=⧽ ???
                                              A⁆
                            A⁆
   A⁃ ⟦㏑e⟧⸢γ⸣(x)         ⧼≜⧽ Aː[t]l
                              A⁅ s′,(rˡ′,rʰ′)
                              A⁃ ␠⟪«where»⟫ ␠ Aː[t]rcl
                                              A⁅ s,(rˡ,rʰ) ⧼=⧽ ⟦e⟧⸢γ⸣(x)
                                              A⁃ s′ ⧼=⧽ \frac{s}{rˡ}
                                              A⁃ rˡ′ ⧼=⧽ ㏑rˡ
                                              A⁃ rʰ′ ⧼=⧽ ㏑rʰ
                                              A⁃ 0   ⧼<⧽ rˡ ≤ rʰ
                                              A⁆
                            A⁆
   A⁃ ⟦⦑relu⦒(e)⟧⸢γ⸣(x) ⧼≜⧽ Aː[t]l
                            A⁅ s′,(rˡ′,rʰ′)
                            A⁃ ␠⟪«where»⟫ ␠ Aː[t]rcl
                                            A⁅ s,(rˡ,rʰ) ⧼=⧽ ⟦e⟧⸢γ⸣(x)
                                            A⁃ s′ ⧼=⧽ ‘❴Aːl@{␠}l@{␠}l
                                                        A⁅ 0    & «⟪if⟫» & rˡ < 0
                                                        A⁃ s    & «⟪if⟫» & rˡ ≥ 0
                                                        A⁆’.
                                            A⁃ rˡ′ ⧼=⧽ rˡ ⊔ 0
                                            A⁃ rʰ′ ⧼=⧽ rʰ ⊔ 0
                                            A⁆
                            A⁆
   A⁆
X⁆
M⁆

M⁅ -10 ≤ x ≤ 2
M⁃ ‹sens›(xx) = 20
M⁃ Aːl
   A⁅ s₁ = 1
   A⁃ s₂ = 1
   A⁃ rˡ₁ = -10
   A⁃ rʰ₁ = 2
   A⁃ rˡ₂ = -10
   A⁃ rʰ₂ = 2
   A⁆
M⁆

\bibliographystyle{plainnat}
\bibliography{local}

\end{document}
\endinput
